\apendice{Especificación de Requisitos}

\section{Introducción}
En este apartado, se van a explicar los requisitos tanto funcionales como no funcionales de la aplicación web realizada, así como los casos de uso y los actores intervienen en ellos.
\section{Objetivos generales}
El objetivo perseguido en este punto es el de implementar una aplicación web capaz de detectar el Parkinson que no requiera demasiados conocimientos para poder utilizarla.
\section{Catálogo de requisitos}
En esta sección, se detallan los requisitos funcionales y no funcionales de la aplicación web realizada.

\subsection{Requisitos funcionales}

\begin{itemize}
	\item \textbf{RF-1 Restricción de acceso a la aplicación:} para poder acceder al sistema, será necesario haberse dado de alta en él.
	\begin{itemize}
		\item \textbf{RF-1.1 Acceso:} los usuarios dados de alta podrán ingresar en la aplicación utilizando su nombre de usuario y su contraseña.
		\item \textbf{RF-1.2 Administración:} habrá un usuario administrador con más acceso que el resto.
	\end{itemize}
	\item \textbf{RF-2 Gestión de usuarios:} el administrador será el único usuario capaz de añadir, modificar y eliminar usuarios.
	\begin{itemize}
		\item \textbf{RF-2.1 Añadir usuario:} Se podrá registrar un usuario en la aplicación.
		\item \textbf{RF-2.2 Modificar usuario:} se podrá modificar alguno de los datos de un usuario existente.
		\item \textbf{RF-2.3 Eliminar ususario:} se podrá eliminar un usuario de la aplicación.
	\end{itemize}
	\item \textbf{RF-3 Subida de datos al servidor:} se podrán subir ciertos datos importantes para detectar el Parkinson.
	\begin{itemize}
		\item \textbf{RF-3.1 Subida de vídeos:} se podrá subir un vídeo al servidor.
		\item \textbf{RF-3.2 Introducción de datos:} se podrá rellenar unos campos con información adicional.
	\end{itemize}
	\item \textbf{RF-4 Visualización del resultado:} se podrá ver si la persona del vídeo subido tiene Parkinson o no.
\end{itemize}

\subsection{Requisitos no funcionales}
\begin{itemize}
	\item \textbf{RNF-1 Usabilidad:} la aplicación tendrá una interfaz amigable para que pueda ser usada sin dificultad.
	\item \textbf{RNF-2 Privacidad:} sólo se podrá acceder a la aplicación si el usuario está dado de alta.
	\item \textbf{RNF-3 Seguridad:} cada usuario dispondrá de una clave que estará encriptada para no ser interceptada por terceros.
	
	
\end{itemize}

\section{Especificación de requisitos}
En esta sección se detalla de forma profunda los casos de uso, qué actores participan en los casos de uso y los correspondientes diagramas.

\subsection{Actores}
Los actores que participan en la aplicación son:

\begin{itemize}
	\item \textbf{Administrador:} dispondrá de todas las funcionalidades con el fin de poder realizar una gestión en la aplicación.
	\item \textbf{Usuario:} podrá subir vídeos al servidor para ser procesados y que se identifique si el usuario tiene Parkinson.
\end{itemize}

\subsection{Diagramas de casos de uso}
