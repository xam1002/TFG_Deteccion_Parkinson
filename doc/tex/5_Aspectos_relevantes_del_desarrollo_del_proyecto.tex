\capitulo{5}{Aspectos relevantes del desarrollo del proyecto}
En este apartado, se detallan los aspectos más relevantes que han ocurrido durante el desarrollo del proyecto, así como la manera en la que el proyecto ha sido realizado para cumplir los objetivos propuestos.

\section{Limpieza de datos}
Como ya se ha explicado en el punto \ref{limpieza}, los datos obtenidos pueden contener ruido que perjudique la tarea que se quiere realizar, en este caso, la clasificación. Por esta razón, es necesario corregirlos de alguna manera.

\subsection{Filtro de Savitzky-Golay}
En el apartado de conceptos teóricos \ref{savgol}, ya se mencionó el filtro de Savitzky-Golay, con el que se podían mejorar las gráficas representando los resultados obtenidos.

Estos resultados contienen ruido tanto por el propio pulso de la mano como por la biblioteca que detecta la mano, por lo que su representación en ocasiones muestra demasiados máximos o mínimos parecidos, confundiendo los valores reales.

\subsection{Distancias} 
Las distancias ya obtenidas no muestran la distancia  real, por lo que es necesario realizar un paso previo antes de calcular el resto de características que dependen de la distancia. 

Dado que la pinza cerrada es el momento en el que menos distancia hay, lo lógico sería pensar que esta distancia es 0, sin embargo, esto no ocurre debido a la localización de los puntos de los dedos. Estos puntos no están justo en la yema de los dedos, donde la pinza se cierra, sino que están en el centro del dedo, por lo que la distancia entre esos puntos nunca será 0, aunque sí muy próxima, es decir, las distancias mínimas nunca serán 0. Al igual que sucede con las distancias mínimas, también las distancias máximas se ven afectadas, y en general, todas las distancias. Por esta razón, se realizan nuevamente las restas de los máximos y mínimos de las distancias anteriormente calculadas para que los nuevos mínimos alcancen valores mucho más cercanos al 0. 

Para comprender esto mejor, se va a suponer que una distancia mínima entre los dos dedos es de 0,051 y una distancia máxima es de 0,376. Al igual que el 0,051 no es el mínimo real, pues lo lógico sería que fuese 0, la verdadera amplitud máxima tampoco sería real, pues debería contar desde el 0. Para obtener la distancia real habría que obtener la diferencia entre el máximo y el mínimo, para conocer la distancia que les separa, es decir, la amplitud de la pinza, que sería 0,325. 

Realmente, todos los puntos no están recogiendo valores reales, sin embargo, dado que sólo interesa recoger la distancia máxima, únicamente se realiza la diferencia entre máximos y mínimos.

\section{Obtención de características}
En este punto se trata de buscar otros datos que caractericen a los vídeos con el fin de poder realizar una clasificación. Habrá datos calculados y datos que ya vienen dados porque no se pueden calcular, como la mano, la edad o el sexo.

\subsection{Amplitud}
Esta característica es la más evidente ya que, generalmente, los pacientes con Parkinson ven dificultad a la hora de abrir y cerrar la mano hasta su punto máximo.

Para su extracción, en primer lugar, se han recogido los máximos y mínimos de la gráfica de los 5 primeros movimientos y de los 5 últimos. Después, estos datos se normalizan para que no afecte la distancia de la mano a la cámara. Esta normalización se obtiene de la siguiente fórmula:

\begin{equation}
	X_{normalizada} = \frac{X_{actual} - X_{mínima}}{X_{máxima}-X_{mínima}}
\end{equation}

Para la extracción de las distancias, tanto la original como la normalizada, tan solo hay que restar el mínimo con el máximo correspondiente. 

\subsection{Tiempo}
Otra de las características que podrían afectar a una persona que tiene Parkinson es cuánto tardaría en realizar el movimiento de pinza, ya que, en un principio, aquellos que padezcan la enfermedad tardarán más.

Para ello, se ha realizado la diferencia entre un máximo y el mínimo correspondiente, pero en este caso, para obtener el número de fotogramas que hay entre ambos puntos, con el fin de utilizar como unidad temporal el número de fotogramas, ya que, a más lentitud, más fotogramas habrá de diferencia y viceversa.

Sin embargo, esta medida resulta no ser suficiente porque no se tiene en cuenta cuánto se ha abierto la pinza, es decir, alguien que abra la pinza lentamente y hasta la mitad tardará el mismo tiempo que alguien que la abra más rápido y hasta el máximo.

\subsection{Velocidad}
Para solucionar el problema anterior, se puede utilizar la velocidad. Esto es, calcular la distancia recorrida por unidad de tiempo, por lo que dependiendo de cuánto se recorra y en cuánto tiempo una persona tendrá más velocidad o no.

En una primera instancia, una persona que padece Parkinson, dada su reducida movilidad, pese a abrir la pinza hasta su punto máximo, lo más seguro es que lo hiciera de forma más lenta que otra persona sin Parkinson.

Esta velocidad se calcula realizando la división de la diferencia normalizada entre un máximo y un mínimo (\textit{i. e.} la amplitud ya calculada), entre el tiempo que les separa a ambos puntos (\textit{i. e.} el tiempo ya calculado).

\subsection{Mano derecha o izquierda}
Como añadido, también se ha tenido en cuenta si la mano es derecha o izquierda. En un principio, podría resultar interesante conocer cuál es la mano que más dificultades tiene a la hora de realizar el movimiento, aunque tampoco resulta fundamental para clasificar. Esta característica no es calculada, sino que viene dada con el vídeo.

\subsection{Sexo}
El sexo de la persona podría servir para clasificar las personas con y sin Parkinson. Como se ha mencionado en la introducción de este documento, el Parkinson es más propenso a aparecer en hombres, por lo que, en primera instancia, podría haber más hombres con Parkinson que mujeres, o lo que es lo mismo, más mujeres sin Parkinson que hombres, lo cual podría ayudar a la hora de clasificar. Al igual que con la mano, esta característica no es calculada, sino que viene dada con el vídeo.

\subsection{Edad}
De forma similar a lo que ocurre con el sexo, el Parkinson es más propenso en personas mayores, de 60 años, por lo que la edad también podría resultar interesante a la hora de clasificar.

Sin embargo, hay que tener cuidado con los ejemplos a la hora de entrenar modelos, ya que el sistema podría aprender que todos los menores de, por ejemplo, 30 años nunca tendrán Parkinson, cuando esto no es cierto. Esto dependerá de la muestra de entrenamiento, ya que si hay pacientes de diversas edades con Parkinson, el modelo realizará la predicción teniendo en cuenta también esos casos. Esta característica tampoco es calculada, también viene dada con el vídeo.

\section{Resultados} \label{resultados}
Dado que el conjunto de datos utilizado para entrenar el modelo no está equilibrado, los resultados no son realmente buenos. Este conjunto de datos no cuenta con demasiados ejemplos de personas sin Parkinson, por lo que el clasificador no es capaz de distinguir correctamente a dos pacientes con y sin Parkinson.

Sin embargo, se ha realizado un estudio de los resultados, utilizando las métricas ya explicadas en el apartado \ref{medidas} para conocer cuál es el clasificador que mejor predice. Además, también se ha realizado un estudio de tiempos, para conocer cuál es el clasificador más rápido.

\subsection{Predicción}
Para realizar la predicción, se han utilizado los clasificadores ya explicados en el apartado \ref{clasificadores}, donde cada columna se corresponde a las siglas de cada clasificador. Se han descartado algunas de las características obtenidas como las amplitudes sin normalizar, la edad o el tiempo ya que empeoraban la clasificación. Además, se ha utilizado una validación cruzada de 2 divisiones y 5 repeticiones, obteniéndose los siguientes resultados:

\begin{table}[h]
	\begin{center}
		\begin{tabular}{ l c c c c c c c }
			\toprule
			\textbf{Medidas} & \textbf{RF} & \textbf{k-NN} & \textbf{AD} & \textbf{NBG} & \textbf{SVM} & \textbf{NB} & \textbf{Dummy} \\ \midrule
			accuracy & 0,775 & 0,8 & 0,712 & 0,719 & 0,812 & 0,812 & 0,812 \\
			Matt. corr. coeff. & -0,001 & -0,025 & 0,048 & -0,045 & 0 & 0 & 0 \\ 
			f1 & 0,87 & 0,889 & 0,821 & 0,824 & 0,897 & 0,897 & 0,897 \\
			ROC AUC & 0,59 & 0,429 & 0,528 & 0,494 & 0,367 & 0,29 & 0,5 \\
			g mean & 0,109 & 0 & 0,295 & 0,098 & 0 & 0 & 0 \\ \bottomrule
		\end{tabular}
		\caption{Tabla con las medidas y clasificadores utilizados.}
		\label{tab:medidas}
	\end{center}
\end{table}

Lo más destacable de estos resultados es el área bajo la curva ROC (\textit{ROC AUC}). Si bien la precisión (\textit{accuracy}) podría ser interesante, en este caso el valor resulta engañoso. Esto se debe a que algunas de estas precisiones realizan una clasificación de la clase mayoritaria, que es Parkinson sí. Dado que el conjunto de datos contiene 6 personas sin Parkinson y 26 con Parkinson, se obtiene una precisión del 0,821 si se realiza una predicción de todo sí:

\begin{equation}
	accuracy = \frac{26 + 0}{26 + 0 + 6 + 0} = 0,8125
\end{equation}

Por esta razón, aunque la mejor precisión sea 0,8125, no se debe considerar como buena, ya que no realizará bien la clasificación cuando se procese un vídeo de una persona sin Parkinson.

El área bajo la curva ROC sí que muestra más información. Por ejemplo, el mejor clasificador en cuanto a este criterio es el Random Forest, con un 0,59, que indica que confunde muchos ejemplos, pero algunos pocos los acierta. También es bastante interesante el resultado que aparece en el clasificador Bayesiano Naive, ya que, pese a ser el peor valor de todos, es decir, 0,29, esto indica que en gran parte confunde los ejemplos de una clase con otra. Si este clasificador se utilizase a la inversa, esto es, cuando prediga sí decir que es no y viceversa, tendría un valor de 0,71 ($1 - 0,29$), que es un valor bastante bueno.

\subsection{Tiempos de entrenamiento}
Se ha realizado un estudio de los tiempos que tardan en entrenar los modelos. Para ello, se han realizado 10 ejecuciones con cada modelo para conocer una aproximación del tiempo que tardan en entrenar, donde se han obtenido los siguientes resultados en segundos:

\begin{table}[h]
	\begin{center}
		\begin{tabular}{ l c c c c c c c }
			\toprule
			\textbf{Ejecuciones} & \textbf{RF} & \textbf{k-NN} & \textbf{AD} & \textbf{NBG} & \textbf{SVM} & \textbf{NB} & \textbf{Dummy} \\ \midrule
			1 & 0,32934 & 0,00397 & 0,00347 & 0,00294 & 0,005 & 0,00505 & 0,00046 \\
			2 & 0,33381 & 0,00391 & 0,00297 & 0,00347 & 0,00561 & 0,00449 & 0 \\ 
			3 & 0,28417 & 0,00319 & 0,00198 & 0,00347 & 0,00546 & 0,00496 & 0,0006 \\
			4 & 0,32095 & 0,00362 & 0,00347 & 0,00347 & 0,00493 & 0,00534 & 0 \\
			5 & 0,26536 & 0,00248 & 0,00198 & 0,00542 & 0,00592 & 0,00496 & 0,0005 \\
			6 & 0,32885 & 0,00149 & 0,00294 & 0,00347 & 0,00545 & 0,00465 & 0 \\
			7 & 0,33331 & 0,00248 & 0,00397 & 0,00298 & 0,00552 & 0,00493 & 0,0005 \\
			8 & 0,32438 & 0,00248 & 0,00298 & 0,00397 & 0,00397 & 0,00496 & 0 \\
			9 & 0,245 & 0,00199 & 0,00252 & 0,00446 & 0,00455 & 0,00298 & 0,0005 \\
			10 & 0,28023 & 0,00244 & 0,00296 & 0,00447 & 0,00528 & 0,00308 & 0,0005 \\ \midrule
			Media & 0,30454 & 0,0028 & 0,00292 & 0,00381 & 0,00517 & 0,00454 & 0,00031 \\ \bottomrule
		\end{tabular}
		\caption{Tabla con los tiempos de entrenamiento de cada modelo con las 10 ejecuciones.}
		\label{tab:tiempos_entrenamiento}
	\end{center}
\end{table}

Viendo los resultados, se puede observar que el clasificador que más destaca frente al resto es el Random Forest. Esto se debe a cómo se construye el clasificador. Dado que Random Forest es un \textit{ensemble}, es decir, que está formado por varios clasificadores, en este caso, árboles de decisión, tiene que realizar la construcción de cada árbol de decisión para construir el clasificador completo. 

En la otra parte está el clasificador Dummy. Como ya se ha comentado, este clasificador realiza una predicción de la clase mayoritaria, es decir, no necesita tomar decisiones internas sobre qué clase asignar a la instancia que se está clasificando, por lo que no requiere apenas tiempo para construir el clasificador.

\subsection{Tiempos de predicción}
También se ha realizado un estudio de los tiempos que tardan en predecir los modelos. Al igual que sucedía con el estudio anterior, se han realizado 10 ejecuciones con cada modelo para conocer una aproximación del tiempo que tardan en predecir, donde se han obtenido los siguientes resultados:

\begin{table}[h]
	\begin{center}
		\begin{tabular}{ l c c c c c c c }
			\hline
			\textbf{Ejecuciones} & \textbf{RF} & \textbf{k-NN} & \textbf{AD} & \textbf{NBG} & \textbf{SVM} & \textbf{NB} & \textbf{Dummy} \\ \hline
			1 & 0,02381 & 0,00502 & 0,00149 & 0,00301 & 0,00397 & 0,00285 & 0,00053 \\
			2 & 0,03476 & 0,00524 & 0,00099 & 0,00244 & 0,00328 & 0,00347 & 0,00099 \\ 
			3 & 0,01885 & 0,00496 & 0,00251 & 0,00347 & 0,00351 & 0,00344 & 0,00035 \\
			4 & 0,02328 & 0,00333 & 0,00249 & 0,00302 & 0,00405 & 0,00313 & 0,0005 \\
			5 & 0,02083 & 0,00248 & 0,00199 & 0,00347 & 0,00343 & 0,00343 & 0,0005 \\
			6 & 0,03274 & 0,00198 & 0,00149 & 0,00248 & 0,00397 & 0,00332 & 0,0005 \\
			7 & 0,03026 & 0,00397 & 0,00198 & 0,00248 & 0,00341 & 0,00347 & 0,0005 \\
			8 & 0,01891 & 0,00347 & 0,00198 & 0,00347 & 0,00298 & 0.00248 & 0,0005 \\
			9 & 0,02527 & 0,00351 & 0,00198 & 0,00297 & 0,00338 & 0,00198 & 0,0005 \\
			10 & 0,03274 & 0,00347 & 0,0015 & 0,00347 & 0,00368 & 0,00341 & 0,0005 \\ \hline
			Media & 0,02614 & 0,00374 & 0,00184 & 0,00303 & 0,00357 & 0,0031 & 0,00054 \\ \hline
		\end{tabular}
		\caption{Tabla con los tiempos de predicción de cada modelo con las 10 ejecuciones.}
		\label{tab:tiempos_pred}
	\end{center}
\end{table}
 
 Nuevamente, el clasificador Random Forest lidera la tabla de tiempos, con mucha diferencia del segundo. Esto se debe por la misma razón que en el entrenamiento, es decir, que se trata de un \textit{ensemble}, pues debe realizar varias predicciones con varios árboles de decisión para poder obtener la mejor. Sin embargo, el tiempo medio es muy inferior al anterior, ya que en general, los clasificadores tardan más en construirse que en realizar la predicción. 
 
 Además, vuelve a destacar el clasificador Dummy, con un tiempo medio muy similar al de entrenamiento. El tiempo que tarda en clasificar es prácticamente insignificante, ya que la única tardanza que tiene es la de asignar a cada instancia la clase mayoritaria.
 
 \subsection{Aplicación web}
Uno de los objetivos propuestos en este trabajo es el de realizar una aplicación web para poder predecir si, dado un vídeo de una mano de una persona, esa persona tiene Parkinson o no. Sin embargo, como se ha explicado en puntos anteriores, esta predicción no va a poder ser realizada de forma correcta, debido a que el conjunto de datos es demasiado malo como para poder entrenar un clasificador capaz de diferenciar de forma aceptable una mano con Parkinson de una que no.

En los resultados de la predicción, se ha destacado el área bajo la curva del clasificador Random Forest, que era superior a la del resto de los clasificadores. Por esta razón, se ha optado por utilizar este clasificador para realizar las predicciones en la aplicación web. No obstante, este modelo puede ser cambiado por el administrador en cualquier momento.

Dado que el software realizado es una aplicación web, este puede ser utilizado desde dispositivos móviles. Por esta razón, se ha implementado una interfaz diferente para pantallas menores a una anchura, por lo que el menú será más cómodo de utilizar en dispositivos de pantallas pequeñas.
