\capitulo{1}{Introducción}
El Parkinson\cite{parkinson} es una enfermedad neurodegenerativa crónica, la cual tiene síntomas como el aumento del tono muscular\cite{tonomuscular} (contracción parcial, pasiva y continua de los músculos) y temblores.

Destaca como trastorno de movimiento, aunque también afecta a la función cognitiva, a la aparición de depresión y dolores, y a la función del sistema nervioso autónomo\cite{sistnervautonomo} (sistema nervioso que controla las funciones involuntarias de las vísceras, como la frecuencia cardíaca o la digestión).

El Parkinson es la segunda enfermedad neurodegenerativa más frecuente, ya que la primera es el Alzheimer. Es más propenso a aparecer en personas mayores de 60 años, sin embargo, podrían comenzar los síntomas desde los 40 años y que la incidencia vaya incrementándose con el paso de los años, aunque, generalmente, esto es más propenso en los hombres.

Es una enfermedad que empeora con el tiempo debido a la destrucción progresiva de las neuronas que están pigmentadas de la sustancia negra\cite{sustancianegra} (ubicadas en una parte del encéfalo), que afecta al sistema nervioso central.

Por ello, un proyecto enfocado a facilitar su identificación puede ayudar al equipo médico encargado y a los pacientes, y con ello, realizar el tratamiento correspondiente cuanto antes para disminuir los daños.

\section{Estructura de la memoria}
La memoria está compuesta por los siguientes apartados:
\begin{itemize}
	\item \textbf{Introducción:} se realiza una breve descripción del cometido del proyecto.
	\item \textbf{Objetivos del proyecto:}
	\item \textbf{Conceptos teóricos:}
	\item \textbf{Técnicas y herramientas:} se indican aquellas técnicas y herramientas que han sido utilizadas para el desarrollo del proyecto.
	\item \textbf{Aspectos relevantes del desarrollo del proyecto:}
	\item \textbf{Trabajos relacionados:} se exponen otros trabajos existentes que guardan relación con este proyecto.
	\item \textbf{Conclusiones y líneas de trabajo futuras:}
\end{itemize}

\section{Estructura de los apéndices}
Los apéndices están compuestos por los siguientes apartados:
\begin{itemize}
	\item \textbf{Plan de Proyecto Software:} se expone la metodología de trabajo utilizada para desarrollar el proyecto.
	\item \textbf{Especificación de Requisitos:}
	\item \textbf{Especificación de Diseño:}
	\item \textbf{Documentación técnica de programación:}
	
\end{itemize}
