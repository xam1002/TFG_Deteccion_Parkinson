\apendice{Plan de Proyecto Software}

\section{Introducción}
En este apartado se expone cómo se ha planificado el proyecto, además de un estudio de viabilidad.
\section{Planificación temporal}
La planificación temporal para la realización de este trabajo se ha realizado utilizando la metodología \textit{Scrum}.\\
\\
Antes de comenzar con los \textit{sprints}, hubo una primera reunión con el objetivo de introducir el tema del proyecto el día 24 de enero de 2022.

\subsection{Sprint 1}
Fecha: 07/02/2022 - 14/02/2022
\begin{itemize}
	\item Instalación de \textit{TeXstudio} y \textit{MikTex}, para poder crear la documentación utilizando \LaTeX{}. Coste estimado: 1. Coste final: 1.
	\item Comprender el código con el cual se van a obtener datos para identificar el nivel de Parkinson. Coste estimado: 3. Coste final: 3.
	\item Comenzar a realizar la documentación del trabajo. Coste estimado 2. Coste final: 2.
\end{itemize}

\textbf{Resumen del sprint:} en este \textit{sprint} se pudo realizar todo sin problema. Coste estimado total: 6. Coste final total: 6.

\subsection{Sprint 2}
Fecha: 14/02/2022 - 21/02/2022
\begin{itemize}
	\item Revisar los máximos de las gráficas, ya que no dan valores claros. Coste estimado: 2. Coste final: 2. 
	\item Aplicar filtrado a las gráficas para corregir los problemas de la biblioteca. Coste estimado: 6. Coste final: 5.
	\item Comentar el código. Coste estimado: 2. Coste final: 2.
\end{itemize}

\textbf{Resumen del sprint:} en este \textit{sprint} se pudo realizar todo sin problema a excepción del filtrado, que no se consiguió uno definitivo. Coste estimado total: 10. Coste final total: 9.

\subsection{Sprint 3}
Fecha: 21/02/2022 - 02/03/2022
\begin{itemize}
	\item Obtener datos de los vídeos y de las gráficas. Coste estimado: 10. Coste final: 10. 
\end{itemize}

\textbf{Resumen del sprint:} en este \textit{sprint} se pudo realizar todo sin problema. Coste estimado total: 10. Coste final total: 10.

\subsection{Sprint 4}
Fecha: 02/03/2022 - 07/03/2022
\begin{itemize}
	\item Realizar un filtrado manual para las gráficas. Coste estimado: 10. Coste final: 8. 
	\item Normalizar los datos. Coste estimado: 5. Coste final: 7.
	\item Investigar un error en la biblioteca que detecta la mano. Coste estimado: 8. Coste final: 6.
\end{itemize}

\textbf{Resumen del sprint:} en este \textit{sprint} se pudo realizar todo sin problema a excepción del filtrado manual, que no cumplió con lo esperado. Coste estimado total: 23. Coste final total: 21.

\subsection{Sprint 5}
Fecha: 07/03/2022 - 14/03/2022
\begin{itemize}
	\item Mejorar el filtrado manual para las gráficas. Coste estimado: 10. Coste final: 10.
	\item Mejorar la extracción de los datos. Coste estimado: 7. Coste final: 7.
\end{itemize}

\textbf{Resumen del sprint:} en este \textit{sprint} se pudo realizar todo sin problema a excepción del filtrado, que no se consiguió solucionarlo en este sprint y tuvo que dejarse para el siguiente. Coste estimado total: 17. Coste final total: 17.

\subsection{Sprint 6}
Fecha: 14/03/2022 - 21/03/2022
\begin{itemize}
	\item Mejorar el filtrado manual para las gráficas (Sprint 5). Coste estimado: 10. Coste final: 10.
	\item Documentar trabajos relacionados de la memoria. Coste estimado: 9. Coste final: 9.
	\item Corregir extracción de datos. Coste estimado: 4. Coste final: 4.
\end{itemize}

\textbf{Resumen del sprint:} en este \textit{sprint} se pudo realizar todo sin problema y finalmente se consiguió un filtrado de Python que cumplía lo esperado. Coste estimado total: 23. Coste final total: 23.

\subsection{Sprint 7}
Fecha: 21/03/2022 - 28/03/2022
\begin{itemize}
	\item Documentar trabajos relacionados de la memoria. Coste estimado: 9. Coste final: 9.
	\item Corregir extracción de datos. Coste estimado: 4. Coste final: 4.
	\item Probar modelos de aprendizaje. Coste estimado: 7. Coste final: 7.
\end{itemize}

\textbf{Resumen del sprint:} en este \textit{sprint} se pudo realizar todo sin problema, aunque los modelos no ofrecían buenas precisiones. Coste estimado total: 20. Coste final total: 20.

\subsection{Sprint 8}
Fecha: 28/03/2022 - 04/04/2022
\begin{itemize}
	\item Continuar documentando trabajos relacionados de la memoria. Coste estimado: 12. Coste final: 12.
	\item Cambiar las columnas de datos. Coste estimado: 5. Coste final: 7.
	\item Probar mejoras en los modelos para mejorar la precisión. Coste estimado: 8. Coste final: 6.
\end{itemize}

\textbf{Resumen del sprint:} en este \textit{sprint} se pudo realizar todo sin problema, aunque no se consiguió ningún modelo nuevo que mejorase la precisión. Coste estimado total: 25. Coste final total: 25.

\subsection{Sprint 9}
Fecha: 04/04/2022 - 18/04/2022
\begin{itemize}
	\item Instalar y comenzar con Flask. Coste estimado: 8. Coste final: 8.
	\item Documentar conceptos teóricos de la memoria. Coste estimado: 8. Coste final: 10.
	\item Organizar y explicar los notebooks. Coste estimado: 5. Coste final: 5.
	\item Extraer otras características de los vídeos para mejorar la precisión de los modelos. Coste estimado: 8. Coste final: 6.
\end{itemize}

\textbf{Resumen del sprint:} en este \textit{sprint} se pudo realizar todo sin problema, pero aunque se mejoró un poco la precisión, no fue suficiente para obtener un buen clasificador. Coste estimado total: 29. Coste final total: 29.

\subsection{Sprint 10}
Fecha: 18/04/2022 - 25/04/2022
\begin{itemize}
	\item Continuar con Flask. Coste estimado: 8. Coste final: 10.
	\item Realizar los cambios en la memoria y documentarla. Coste estimado: 5. Coste final: 5.
	\item Cambiar las métricas del entramiento de modelos. Coste estimado: 5. Coste final: 3.
	\item Crear los requisitos funcionales y no funcionales de la aplicación, junto a los diagramas de casos de uso. Coste estimado: 8. Coste final: 8.
\end{itemize}

\textbf{Resumen del sprint:} en este \textit{sprint} se pudo realizar todo sin problema. Coste estimado total: 26. Coste final total: 26.

\subsection{Sprint 11}
Fecha: 25/04/2022 - 09/05/2022
\begin{itemize}
	\item Implementar el funcionamiento de la aplicación web. Coste estimado: 8. Coste final: 10.
	\item Añadir técnicas y herramientas en la memoria. Coste estimado: 8. Coste final: 8.
	\item Corregir requisitos y casos de uso. Coste estimado: 5. Coste final: 5.
	\item Documentar los casos de uso. Coste estimado: 8. Coste final: 8.
	\item Diseñar las interfaces de las ventanas de la aplicación. Coste estimado: 8. Coste final: 6.
\end{itemize}

\textbf{Resumen del sprint:} en este \textit{sprint} se pudo realizar todo sin problema. Coste estimado total: 37. Coste final total: 37.

\subsection{Sprint 12}
Fecha: 09/05/2022 - 16/05/2022
\begin{itemize}
	\item Cambiar apariencia de la aplicación. Coste esperado: 8. Coste final: 12.
	\item Añadir la opción de modificar usuario para usuarios sin privilegios. Coste esperado: 5. Coste final: 5.
	\item Comenzar con la documentación técnica de programación. Coste esperado: 8. Coste final: 6.
	\item Medir el tiempo que tardan en predecir los modelos. Coste esperado: 8. Coste final: 3.
\end{itemize}

\textbf{Resumen del sprint:} en este \textit{sprint} se pudo realizar todo sin problema. Coste estimado total: 29. Coste final total: 26.

\subsection{Sprint 13}
Fecha: 16/05/2022 - 23/05/2022
\begin{itemize}
	\item Cambiar más aspectos de la apariencia de la aplicación. Coste esperado: 13. Coste final: 13.
	\item Realizar estudio de tiempos. Coste esperado: 8. Coste final: 8.
	\item Buscar servidor en la nube para alojar la aplicación web. Coste esperado: 8. Coste final: 15.
	\item Añadir certificado SSL al servidor. Coste esperado: 8. Coste final: .
\end{itemize}

\textbf{Resumen del sprint:} en este \textit{sprint} se pudo realizar todo sin problema a excepción de encontrar el servidor y añadir el certificado SSL. Esto se debió a la dificultad de encontrar una página de servidores que pudiera alojar el proyecto. El certificado SSL queda a la espera de encontrar un servidor definitivo. Coste estimado total: 37. Coste final total: .

\subsection{Sprint 14}
Fecha: 23/05/2022 - 30/05/2022
\begin{itemize}
	\item Utilizar Heroku como servidor en la nube para alojar la aplicación web. Coste esperado: 8. Coste final: .
	\item Añadir certificado SSL al servidor (Sprint 13). Coste esperado: 8. Coste final: .
	\item Realizar especificación de diseño. Coste esperado: 8. Coste final: .
	\item Realizar la documentación de la memoria. Coste esperado: 8. Coste final: 10.
	\item Continuar con la documentación técnica de programación. Coste esperado: 8. Coste final: .
\end{itemize}

\textbf{Resumen del sprint:} en este \textit{sprint} sólo se pudo realizar la documentación de la memoria y parte de la especificación de diseño debido a la falta de tiempo. También hubo problemas la hora de arrancar la aplicación en Heroku, ya que aparecían errores. Nuevamente, el certificado SSL queda a la espera de encontrar un servidor definitivo. Coste estimado total: 40. Coste final total: .

\section{Estudio de viabilidad}

\subsection{Viabilidad económica}

\subsection{Viabilidad legal}
