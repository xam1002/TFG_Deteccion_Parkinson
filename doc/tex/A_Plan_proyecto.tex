\apendice{Plan de Proyecto Software}

\section{Introducción}

\section{Planificación temporal}
La planificación temporal para la realización de este trabajo se ha realizado utilizando la metodología \textit{Scrum}.\\
\\
Antes de comenzar con los \textit{sprints}, hubo una primera reunión con el objetivo de introducir el tema del proyecto el día 24 de enero de 2022.

\subsection{Sprint 1}
Fecha: 07/02/2022 - 14/02/2022
\begin{itemize}
	\item Instalación de \textit{TeXstudio} y \textit{MikTex}, para poder crear la documentación utilizando \textit{LaTeX}. Coste estimado: 1. Coste final: 1.
	\item Comprender el código con el cual se van a obtener datos para identificar el nivel de \textit{Parkinson}. Coste estimado: 3. Coste final: 3.
	\item Comenzar a realizar la documentación del trabajo. Coste estimado 2. Coste final: 2.
\end{itemize}

\subsection{Sprint 2}
Fecha: 14/02/2022 - 21/02/2022
\begin{itemize}
	\item Revisar los máximos de las gráficas, ya que no dan valores claros. Coste estimado: 2. Coste final: 2. 
	\item Aplicar filtrado a las gráficas para corregir los problemas de la biblioteca. Coste estimado: 6. Coste final: 5.
	\item Comentar el código. Coste estimado: 2. Coste final: 2.
\end{itemize}

\subsection{Sprint 3}
Fecha: 21/02/2022 - 02/03/2022
\begin{itemize}
	\item Obtener datos de los vídeos y de las gráficas. Coste estimado: 10. Coste final: 10. 
\end{itemize}

\subsection{Sprint 4}
Fecha: 02/03/2022 - 07/03/2022
\begin{itemize}
	\item Realizar un filtrado manual para las gráficas. Coste estimado: 10. Coste final: . 
	\item Normalizar los datos. Coste estimado: 5. Coste final: 7.
	\item Investigar un error en la biblioteca que detecta la mano. Coste estimado: 8. Coste final: .
\end{itemize}

\section{Estudio de viabilidad}

\subsection{Viabilidad económica}

\subsection{Viabilidad legal}
