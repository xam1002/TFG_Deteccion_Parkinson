\capitulo{4}{Técnicas y herramientas}
\section{Herramientas de desarrollo}
\subsection{Anaconda}
Anaconda es una distribución libre utilizada para los lenguajes de programación Python y R con el objetivo de realizar aprendizaje automático y ciencia de datos. Entre sus paquetes se encuentra Jupyter Notebook.

\subsection{Jupyter Notebook}
Jupyter Notebook es un entorno de programación para Python basado en la web. Se pueden crear varios notebooks con celdas de código o texto para conseguir una estructura limpia y ordenada.

\section{Bibliotecas}
\subsection{OpenCV}
Es una biblioteca de Python utilizada para visión artificial y es considerada la más popular. Tiene diversos usos, entre ellos destacan el reconocimiento de objetos y la detección de movimiento.

\subsection{Numpy}
Es una biblioteca de Python utilizada para realizar operaciones matemáticas. También se usa para crear vectores y matrices grandes multidimensionales.

\subsection{Pandas}
Es una biblioteca de Python utilizada para el análisis de datos. Mediante \textit{dataframes} se pueden recoger datos de hojas de cálculo o bases de datos, almacenarlos para ser tratados en el código y después guardarlos de nuevo.

\subsection{Mediapipe}
Es una biblioteca de Pyhton utilizada para visión artificial. Una de las funcionalidades que tiene es la de reconocer y enumerar con puntos una mano.
